%nomencl宏包说明
\documentclass{ctexart}
 \usepackage[refeq,refpage]{nomencl}
 \usepackage{geometry}
 \usepackage{fancyhdr}
 \usepackage{makecell}
 \usepackage{enumitem}
 \usepackage[dvipsnames,svgnames,table]{xcolor}
 \usepackage{listings}
 \usepackage{framed}
 \usepackage{graphicx}
 \usepackage[colorlinks,linktocpage,linkcolor=blue]{hyperref}
 \usepackage{titletoc}
 \usepackage{amsmath}
 \usepackage{cprotect}
 \usepackage{shortvrb}
 


 %定义所需要的颜色
 \definecolor{pack}{RGB}{199,49,124}		
 \definecolor{mBackground}{RGB}{255,248,220}
 \definecolor{maincolor}{RGB}{0,32,245}
 
 \setlength\fboxsep{1.5pt}
 
 \setlist[description]{font= \color{pack}\sffamily\bfseries}	%设置 descripti 的样式
 \MakeShortVerb| 
 \newcommand{\mcode}[1]{\noindent \caprotect \fbox{#1} 
 }
 
% \lstset{	
% 	firstnumber=1,
% 	frame = single,
% 	breaklines = true,
% 	xleftmargin=0pt,
% 	xrightmargin=0pt,
% 	basicstyle=\small,
% 	showtabs=false,
% 	tabsize=4, 
% 	backgroundcolor=\color{mBackground},
% 	numberstyle=\itshape,
% 	framesep=1pt,
% }

\lstset{
	basicstyle=\small\ttfamily,%整体格式设置
	columns=fullflexible,%设置字符列为非等宽
	breaklines=true,%自动换行
	% frameround=ttff,%边框四角弧度,t则弧
	frame=tb,%边框样式
	backgroundcolor=\color{mBackground},%背景颜色
	framerule=1.5pt,%边框线宽度
	rulecolor=\color{green!30},%边框线颜色
	% framesep=.5em,%边距
	numbers=left,%左边显示数字
	% numbersep=0em,
	xleftmargin=2em,
	keywordstyle = \color{blue},
	identifierstyle = \color{blue!70},
	framexleftmargin=2em,
	numberstyle={\sffamily\footnotesize\color{teal}}
}
 
 %定义新的颜色输出
 \newcommand{\textbl}[1]{\textbf{\textcolor{maincolor}{#1}}}      %蓝色加粗字体
 \newcommand{\tbt}[1]{\colorbox{lightgray!70}{#1}}		%定义行内的命令显示			
 \newcommand{\txr}[1]{\textcolor{red}{#1}}	
 \newcommand{\txg}[1]{\textcolor{teal}{#1}}
 
 %设置页面
 \geometry{left=2.5cm,right=1.5cm,top=2.5cm,bottom=2.5cm}
 %设置页眉页脚
 \pagestyle{plain}
 
 %设置标题样式
 \ctexset{
 	section = {
 		format+ = \zihao{-4} \heiti \centering \color{blue},
 		name = {,、},
 		number = \chinese{section},
 		beforeskip = 1.0ex plus 0.2ex minus .2ex,
 		afterskip = 1.0ex plus 0.2ex minus .2ex,
 		aftername = \hspace{0pt}
 	},
 	subsection = {
 		format+ = \zihao{5} \heiti \raggedright,
 		% name={\thesubsection、},
 		name = {,、},
 		number = \arabic{subsection},
 		beforeskip = 1.0ex plus 0.2ex minus .2ex,
 		afterskip = 1.0ex plus 0.2ex minus .2ex,
 		aftername = \hspace{0pt}
 	}
 }
 
\makenomenclature
 \begin{document}
 
\title{使用 nomencl 制作术语表}	
\date{\today} 
\author{种田er}
\maketitle
 
\section{制作一个例子}

\par 
首先,需要将 |\makenomenclature| 放命令在你的前言里,事实上,在测试中发现放在导言区和放在正文都可以。然
后输入一个引用的公式。而后使用 |\nomenclature{symbol}{description}| 命令列举所需要的术语,包括符号、以及描
述,最后使用 |\printnomenclature| 命令在合适的地方打印出术语表。
\begin{equation}
	a=\frac{N}{A}
\end{equation}%
\nomenclature{$a$}{The number of angels per unit area\nomrefeq}%
\nomenclature{$N$}{The number of angels per needle point\nomrefpage}%
\nomenclature{$A$}{The area of the needle point}%
The equation $\sigma = m a$%
\nomenclature{$\sigma$}{The total mass of angels per unit area\nomrefeqpage}%
\nomenclature{$m$}{The mass of one angel}


\par 
想要打印出术语表,\txr{首先在命令行切换到当前路径下,使用 pdflatex 或者 xelatex 进行一次编译}。然后会产生一个 
\txr{*.nlo} 文件,随后在命令行中输入 |makeindex filename.nlo -s nomencel.ist -o filename.nls| 命令。此命令会根据生成的 *.nlo文件
会生成一个  \txr{*.nls} 文件,该文件会内部会自动将术语放入 \txr{thenomenclature} 
环境,再次编译则会预定的显示方式进行显示,例如见 公式、页数等。

%nlo 文件中的代码
\noindent \textbf{*.nlo 文件中的代码}
\begin{lstlisting} 
\nomenclatureentry{a$a$@[{$a$}]\begingroup The number of angels per unit area\nomrefeq\nomeqref 
{1}|nompageref}{1}
\nomenclatureentry{a$N$@[{$N$}]\begingroup The number of angels per needle point\nomrefpage\nomeqref 
{1}|nompageref}{1}
\nomenclatureentry{a$A$@[{$A$}]\begingroup The area of the needle point\nomeqref {1}|nompageref}{1}
\nomenclatureentry{a$\sigma$@[{$\sigma$}]\begingroup The total mass of angels per unit area\nomrefeqpage\nomeqref 
{1}|nompageref}{1}
\nomenclatureentry{a$m$@[{$m$}]\begingroup The mass of one angel\nomeqref {1}|nompageref}{1}
\end{lstlisting} 


%nls文件中的代码
\noindent \textbf{*.nls 文件中的代码}
\begin{lstlisting}
\begin{thenomenclature} 
\nomgroup{A}
\item [{$\sigma$}]\begingroup The total mass of angels per unit area\nomrefpage\nomeqref {1}\nompageref{1}
\item [{$A$}]\begingroup The area of the needle point\nomeqref {1}\nompageref{1}
\item [{$a$}]\begingroup The number of angels per unit area\nomrefeq\nomeqref {1}\nompageref{1}
\item [{$m$}]\begingroup The mass of one angel\nomeqref {1}\nompageref{1}
\item [{$N$}]\begingroup The number of angels per needle point\nomrefpage\nomeqref {1}\nompageref{1}
\end{thenomenclature}
\end{lstlisting}

%编译的方式
\noindent \textbf{编译的方式}
\begin{lstlisting}
xelatex filename
makeindex filename.nlo -s nomencel.ist -o filename.nls
xelatex filename
\end{lstlisting}

\section{运作方式与 Makeindex 命令}
\subsection{运作方式}
\par 
事实上,|\makenomenclature| 将会引导 \LaTeX 打开一个 *.nlo 文件,并且将 |\nomenclature| 后的术语与描述以一个  
合适的方式写如 *.nlo 文件中。
\par 
下一步就是调用 |Makeindex| 命令,并且将 *.nlo 作为输入文件,使用 nomencl.sty 作为样式文件,最后将 *.nls 文件
作为输出文件。
\par 
有了 *.nls 文件,再次编译,即可打印出术语表,\txr{事实上,在测试中,发现只需要有了 *.nls 
	文件再编译即可打印出术语表,并且如果打开 *.nls 文件中并且按照相应格式添加 item 
	的话也会打印出新添加的术语,即起决定作用的是 *.nls 文件。}

\subsection{Makeindex 命令}
\par
Makeindex 是一个运行命令行下的程序,它的功能是对 \TeX 编译程序产生的索引进行排序、归并,得到用于输出的
索引表,其最基本的用法是
\begin{lstlisting}
makeindex filename.idx
\end{lstlisting}
这样会得到排序整理后的 filename.ind 文件,文件后缀 .idx 可以省略。关于 Makeindex 一些常用的选项

\bigskip
\begin{tabular}{rl}
    -o \textit{<ind>}     &     设置输出文件为 \textit{<ind>}。这一项默认是输入文件的文件名加上后缀名.ind。\\
    -s \textit{<sty>}     &		 设置格式未见为\textit{<sty>} 。这一项默认为空,格式一般以 .ist 为文件后缀。   \\
    -t \textit{<log>}    &		设置日志文件为 \textit{<log>}。日志文件记录了 Makeindex 处理的词条汇总和错误信息。
 \end{tabular}

\noindent \txr{关于更多 makeindex 命令的 style,可以参考 《\LaTeX 入门》中的相关章节。}

%主要顺序与可选选项
\section{主要命令与可选选项}

\subsection{主要命令}
\noindent |\nomenclature[⟨prefix⟩]{⟨symbol⟩}{⟨description⟩}|
\par 由于|\nomenclature| 的扫描方式,所以无需通过 |\protect| 方式保护脆弱命令。但也必须确保在第一个参数之前或
在第一个参数与第二个参数之间不包含任何字符,尤其是没有换行符(即使使用$\%$)。所以
|\nomenclature {$x$}%{描述}| 并不起作用。
\begin{description}
\item [prefix] 用来微调术语表的打印顺序,默认的打印顺序为按照列举的顺序按数字排列,也可按照字符或者 ASCII 
码进行排序。我们以一个例子来看排列顺序的规则。假设我们有 |$~Ab$$~aa$, $\Ab$, $\aa$, |\\ 
|$Ab$, $aa$, Ab, aa |这8个需要排列的术语。如果使用默认的选项,那么所有的术语都会被视作字符。并且你会得到
|$\aa$, $\Ab$, $aa$, $Ab$, $~aa$, $~Ab$, aa, Ab| 的排列顺序。如果在加载宏包的时候加入了 noprefix 的选项,那么
你会得到 |$Ab$, $\Ab$, $\aa$, $aa$, $~Ab$, $~aa$, aa, Ab| 的排列顺序,前 6 者被视作符号并且按照 ASCII 码进行
排序,而对于 Ab 和 aa ,由于 A 有更小的 ASCII 码所以排序更靠前。
\item [symbol]  是你需要列举的术语。
\item [description] 是你需要列举术语的描述。
\end{description}

\subsection{可选选项}
\begin{description}
\item[refeq] 如果加入此选项,在术语表中你会看到 see eqution(number) 的样式。
\item[norefeq] 与前者相反,无 see eqution(number) 的样式。
\item[refpage] 如果加入此选项,在术语表中将会看到 page(number) 的样式。
\item[norefpage] 与前者相反,无 page(number) 的样式。
\item[prefix] 在前面的举例中已经有所讲述。
\item[noprefix]  与前者相反。
\item[stdsubgroups] 根据索引的含义将索引条目分成几组。
\item[stdsubgroups] 与前者相反。
\item[cfg] 默认情况下,nomencl.cfg 会被加载,如果存在一个另外的 cfg = filenmae 则会使用它。
\item[nocfg] cfg 文件不会被加载。
\item[intoc] 将术语表加入到目录中去。
\item[notintoc]  不将术语表插入到目录中。
\item[tocbasic] 默认选项,使用 KOMA 脚本包中的 tocbasic 宏包进行 TOC 处理。
\item[notocbasic] 	与前者相反。
\item[compatible] 在兼容模式下运行。较旧的 tex 
文件可能需要选择此选项才能进行编译。在最新版本的nomencl中,命令
|\makeglossary| 和 |\printglossary| 被替换为 |\makenomenclature|和 
|\printnomenclature|。选择此选项将重新定义旧命令,但将失去与其他词汇表宏包的兼容性。
\item[noncompatible] 不以兼容模式运行。
\item[nomentbl]  以 nomentbl 样式打印术语。
\item[nonomentbl] 与前者相反。
\end{description}

\subsection{nomentbl 样式} 
\par 如果使用 nomentbl 样式的话,| \nomenclature| 将有四个参数,即
\begin{lstlisting}
\nomenclature[<prefix>]{<symbol>}{<description>}{<units>}{<note>}
\end{lstlisting}
前面 3 者与原来一致,包含在 siunitx 宏包的 |\si| 命令中,而 |\note| 内部添加在描述中的任意注释。

%测试用的代码
\setlength{\nomitemsep}{0.2cm}
\renewcommand{\nomentryend}{.}
\renewcommand{\eqdeclaration}[1]{见公式:(#1)}
\renewcommand{\pagedeclaration}[1]{见第 (#1) 页}
\renewcommand{\nomname}{术语表}
\section{定制}

定制主要是修改术语表与描述之间的间距,各种显示的样式。
\begin{itemize}
	\item 可以在 |\printnomenclature| 命令后面的方括号中加入距离可以修改术语和描述之间的距离,\txr{米制距离不
		被允许。}
	\item 如果不喜欢这样的样式,则需要重新定制 nomenclature 环境。
	\item 如果想要修订术语表的名称,则可以使用,|\renewcommand{\nomname}{List of Symbols}| 命令。 
	\item 如果想要修改各个术语项目之间的间距则需要使用 |\setlength{\nomitemsep}{space}| 命令。
	\item 如果想要修改术语表中公式的引用样式,则可以使用 |\renewcommand{\eqdeclaration}[1]{style #1}| 命令。
	\item 如果想要修改页码引用的样式,则可以使用 |\renewcommand{\pagedeclaration}[1]{style#1}| 命令。
	\item 想要修改术语表的名称,则需要使用 |\renewcommand{\nomname}{name}| 命令,该格式用 section 的
	格式进行控制,与 section 的格式相同。
\end{itemize} 

\par 更多修订方式可以参考 nomencl 宏包。

\section{总结}
想要打印出术语表,重要的是在第二步中使用命令行的 |makeindex| 命令输出一个*.nls 文件,有了此 nls 文件并且
加入 |\printnomenclature[0.5in]| 命令再次编译就可以得到想要的术语表,当然可以在你想要的部分加入 print 命令以
在需要的地方显示术语表。

\printnomenclature[0.5in]
\end{document}